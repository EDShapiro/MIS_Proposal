\documentclass{article}
\usepackage{amsmath}
\usepackage{amssymb}
\usepackage{gensymb}
\usepackage{hyperref}
\usepackage{listings}
\usepackage{graphicx}
\usepackage[export]{adjustbox}
\usepackage{enumitem}
\usepackage{subcaption}
\usepackage[margin=1.25in]{geometry}
\linespread{2.0}
\graphicspath{ {c:/Users/EvanD/OneDrive/Pictures/Thesis/}}
\title{Integrated Science Thesis Proposal}
\author{Evan Shapiro \\ Master's of Integrated Science, University of Colorado Denver}
\begin{document}
<<<<<<< HEAD

%<<<<<<< HEAD

=======
>>>>>>> 7743f98897ec0e1a10dda70ead44275893739ee7

\begin{abstract}
Ion diffusion at ITER scales must beGlobal gyrokinetic models have been shown to accurately predict the ion diffusivity by incorporating microturbulence into the models, $\chi_i$.
Extreme-scale, fixed-flux supercomputing simulations are beginning
to simulate modes of operation relevant to next-generation(ITER,DEMO)
reactors. Using a surrogate model to reduce the computational expense of XGC simulations, I conduct a predictive scan
in $\rho^{-1}$ , to ascertain
whether or not the ion diffusivity $\chi_i$ scales in a Bohm or gyro-Bohm
fashion, and analyze the sensitivity of $\chi_i$ to perturbation in the heating
model.% in the $XGC$ model.
\end{abstract}

\maketitle
\tableofcontents

\maketitle
\tableofcontents


\section{Introduction}
\subsection{Magnetically Confined Fusion}
After Hans Bethe developed the theory for nuclear fusion in the sun, showing that only hydrogen to helium nuclear reactions can account for the energy production by the sun  \cite{Bethe}, fusion research to produce these processes began globally. Due to increasing computational power, as well as international  research collaborations, the 21st century has seen a resurgence in interest fusion energy as serious contender for sustainable energy, the most norable being the international thermonuclear experimental reactor (ITER), which will resolve many of the issues that currently stand in the way of a working fusion reactor.\\
Nuclear fusion is appealing as an energy source due to the net positive energy release that occurs when two light elements collide with enough energy to cause the transfer of a nucleon from one nucleus to the other. In the deuterium-tritium reaction a proton from the deuterium atom is fused to the tritium atom, yielding a release of a neutron, and $\alpha$ particle, and the release of $17.6 MeV$ of binding energy. $14.1 MeV$ of this binding energy is given to the neutron, and it is this energy that a fusion reactor taps into to power a steam turbine. The nuclear equation for dueterium-tritium reaction is presented below, along with an illustration:
$$
D+T \rightarrow \alpha + n + 17.6 MeV
$$
 \cite{J_Friedberg:1} .\\
\\
\begin{figure}[h]
	\centering
	\includegraphics[width=0.5\columnwidth]{D_T_MP}
	\caption{Deuterium-Tritum Reaction Image Courtesy of Max-Planck Institute}
\end{figure}
\\
Creating a sustained reaction of a 50-50 $\%$ Deuterium-Tritium mixture requires an input of $70KeV$, corresponding to a temperature on the order of $100*10^6$K. It is this high temperature environment that leads to the complete ionization of the deuterium-tritium mixture, resulting in a plasma.\\
More detail will be included below - just need to get ideas down.\\
%Is this enough detail on power balance? Should I go into more detail on what the sources and sinks are?
Due to the engineering requirements for magnetic pressure, plasma confinement time, and plasma temperature, with the complete explanation of these requirements being beyond the scope of this proposal, a toroidal geometry is the accepted geometry for a fusion reactor. The basic design requirements are as follows. The plasma is contained within the inner cylinder of a toroid, with the first wall of the cylinder being protected by a strong magnetic field that shields the wall from thermal loading from the $14.1 MeV$ neutrons, Bremsstrahlung heat loss, and heat conduction. Inside the first wall is the region called the blanket and shield, which is the region of heat exchange for powering an external steam turbine, and provides a tritium breeding ground that sustains the supply of tritium required for the fusion reaction. Outside of this region is the a shield wall which prevents the escape of any radioactive neutrons or gamma particles. The final layer of the toroid contains the superconducting toroidal solenoid, which generates the $10-15 [T]$ magnetic field inside of the fusion reactor, which again is responsible for confining the ionic and electron components of the plasma inside the reactor. Figures 1 shows the generic theoretical cross-sectional area of toroidal reactor, while figure 2 shows the cross-sectional area of the ARIES-AT tokamak design.\\
\\
\begin{figure}[h]
\begin{subfigure}[t*]{0.5\textwidth}
\includegraphics{Basic_Reactor}
\caption{Generic toroidal fusion reactor showing the plasma, blanket-and-shield, and magnets. Image Source J. Friedberg \cite{J_Friedberg:1} }
\end{subfigure}
\begin{subfigure}[t*]{0.5\textwidth}
\includegraphics[width = 0.6\columnwidth]{ARIES_AT}
\caption{Cross section of the ARIES-AT power core configuration (courtesy of F. Najmabadi). \cite{J_Friedberg:1} }
\end{subfigure}
\end{figure}
%%I need to explain the geometry of a Tokamak more clearly.
%%
\\
Not shown in the picture is the center, or donut hole, of the toroidal reactor. In current reactors, there is an additional magnet in this region that produces a poloidal magnetic field, which when combined with the toroidal magnetic fields produces the magnetic field lines, as seen in figure 3. Due to the Lorentz force, this configuration of magnetic field lines manages to capture the majority of the diffusing ions and electrons in a helical tractory around the magnetic fields lines due to to the Lorentz force on the charged particles.
\subsection{Diffusion, Neoclassical Theory, and Scaling Laws}
Maintaining a viable, power producing fusion reactor requires that power inputs and outputs be balanced within the reactor.
 This power balances requires an accounting of all of the power sinks and sources available to the plasma, as well as an accounting of all the transport mechanisms within the plasma. 
While we are using a 5-D gyrokinetic model in our simulation work, it can be illustrative to use the 0-D conservation of energy equation to demonstrate the basic principle of power balance. 
The 0-D fluid conservation of energy equation is given by:
$$
\frac{3}{2}\frac{\partial p}{\partial t} + \frac{3}{2}\nabla \cdot p \vec{v} + p\nabla \cdot v + \nabla\cdot \vec{q} = S
$$
where the internal energy $U$ has been substituted for $\frac{3}{2}p$\cite{J_Friedberg:1}.\\
Let's analyze this equation: The first term accounts for the variation of energy flowing into and out of the system with respect to time, the second term accounts for energy lost or gained due to convection of heat out of the system, the third term accounts for energy losses due to compression and expansion of the plasma within a reactor, the 4th term account for heat losses due to diffusion and conduction, and $S$ accounts for the sources and sinks available to the plasma. The sources are the $\alpha$ particle heating due to the D-T nuclear reaction and any auxiiliary heating being supplied to the plasma, while the sinks are the neutrons escaping the plasma into the blanket surrounding the reactor, and Bremsstrahlung radiation.\\
%%
\\
Assuming that the power production in the reactor has reached a steady state the time term drops out, as well as the convection and compression terms, leaving the 0-D energy conservation equation:
$$
\nabla\cdot \vec{q} = S.
$$
This equation makes intuitive sense. In equillibrium, if the diffusive processes are matched by the power sources sources and sinks, then a power balance will be achieved in the equation. If the diffusion of heat out of the system increases,  $\alpha$ particle heating has to increase to maintain power balance in the reactor, decreasing the efficiency of the reactor. \\ %Is this the correct interpretation of this equation?\\
%%
%%
Maintaining a plasma reactor with a viable power output requires that the transport of heat out of the plasma be balanced by the alpha heating power of the plasma. Characterizing the heat conduction parameter, or ion heat diffusivity $\chi_i$, in a plasma is current research, \cite{Yas_Ido} \cite{choong-seock_chang_et_al_partnership_2017} as it is the primary heat loss mechanism in a tokamak reactor. Per Friedberg “In a plasma there are three important types of transport: heat conduction, particle diffusion, and magnetic field diffusion. Of these, heat conduction is the most serious loss mechanism... \cite{J_Friedberg:1}." \\

Classic and neoclassical diffusion model the ion heat diffusivity using a random walk model of the ions and electrons as they move through their gyrokinetic trajectory around the magnetic field lines in the toroidal reactor, interacting with other charged species along their trajectory, and imparting thermal energy. Neoclassical transport theory refers to the diffusion process within a toroidal geometry, which generates for a 2.4 fold increase in ion heat diffusion.\\
The classical and neoclassical models are unable to account for the observed, anomalously large transport of energy and particles across the confining magnetic field. It is assumed that that the anomolous ionic thermal diffusivity is a function of various local dimensionless quantities, including $\beta$ the ratio of plasma to magnetic  pressure, $\rho^*$, which is explained below.  Current research \cite{Yas_Ido} focuses on plasma microturbulence caused by steep ionic thermal gradients as the driving mechanism for this anomalous heat transport, and incorporates scaling arguments in $\rho^*$ to make predictions using available data from smaller scale fusion reactors to predict how ion heat diffusivity will scale to ITER levels. Scaling laws are laws used in fluid dynamics that are allowed to be made under dynamic and geomtric similarity situations, thus since the dynamics and geometries between tokamaks is nearly identical, scaling laws are useful when experimental validation is not available.\\

We define $\rho^*$ to be $\rho^* = \rho_i/a$ where $\rho_i$ is the gyrokinetic radius of the ion as it moves around the magnetic field in its trajectors through the fusion reactor, and $a$ is the minor radius of the tokamak, or the radius of the inner cross-sectional area of the tokamak. The reason that $\rho^*$ is interesting is because "Existing experimental devices can match all of these transport relevant dimensionless parameters expected in a reactor scale with the exception of $\rho^*$ \cite{McKee_Et}." The inability to set the parameter $rho^*$ in a lab experiment, and the advanced modeling and simulation capabilities of current tech, make a new study on ionic thermal diffusivity an ideal topic for a master's thesis\\

Thus, the stated purpose of this thesis is to determine how the ionic thermal diffusivity scales with a scan in $\rho^*$ up to ITER scales, to determine whether the diffusion becomes Bohm or gyro-Bohm like. Bohm diffusion indicates that the diffusion increases linearly with temperature, and is undesireable, while gyro-Bohm diffusion indicates that the  diffusion scales sublinearly with temperature, and is desireable.
%%Add More Detail Above

%%This may be redundant with later material

We will use available experimental fusion reactor data, and implement surrogate modeling techniques to reduce the computational expense of running a simulation of 5-D  gyrokinetic model  of a plasma within a tokamak reactor, which is the geometr of the ITER reactor. The outcome of this project will be a pdf of the ion diffusion, with uncertainties. Once this is accomplished, if time permits, we will perform a %push forward 
comprehensive sensitivity analysis of the heating model.

\subsection{Gyrokinetic Equations}
The complete derivation of the governing set of equations of multi-scale physics in a tokamak reactor is well beyond the scope of this paper. However, the important physics that are described by the governing equations should be covered for clarity and reference. The governing set of equations of a plasma are called the Vlasov-Boltzmann equations and are provided below: 
$$
$$
The first equation is the Boltzmann convection-collision equation  modified for an electromagnetic system. Since the majority of the interactions in a plasma are Coulombic interactions, and they dominate the characteristics of the plasma, the collision term from the full Boltzmann eqution is ignored. The Vlasov equation decribes how the plasma is going to travel due to external forces acting on the plasma.
\subsection{XGC Code \& Heating Models}
\section{Experimental Metholodogy}

\section{Prediction under Uncertainty}
%
% Overview of the Problem
%

Prediction under uncertainty is often an expensive and complicated process\cite{odensiam1,odensiam2}.
The tradeoff between the computational cost of complex models and the loss of predictive accuracy 
associated with simpler models may be addressed by emerging multifidelity UQ approaches\cite{ngmulti,Pehermulti}.

In magnetic confinement fusion, there is an important, clearly defined set of prediction scenarios, corresponding 
to modelling future reactor(ITER/DEMO) performance.  Even a ``high-fidelity'', extreme-scale model such as XGC, still
has a parameter space of large enough dimension to make a brute-force, 
sampling-based predictive process
impossible.  
%The $N^{-1/2}$ convergence, where $N$ is the number of samples of the quantity of interest(QOI), is infeasible given the requirements on computational resources. 
The traditional approach to prediction, with or without extrapolation, is to sample the model parameter input space
$\theta=\theta_1,\theta_2, \ldots, \theta_d$, evaluate the model, and return a quantity of interest (QoI) $Q(\theta)$.
Most realistic QoI maps are nonlinear in the QoI map(even if the governing PDE is linear), 
so the probability distribution function(PDF) of $Q(\theta)$ will
have to be estimate in a non-parametric way, typically
 by kernel density estimation, or by computing empirical 
statistics from samples.
The mean-integrated squared error
of the approximate PDF converges with a rate of $\mathcal{O}(N^{-2/(d+4)})$, where $N$ is the number of samples of the model
and $d$ is the dimension of the input space.  As each sample typically involves a PDE solve and subsequent post-processing,
this process quickly becomes exorbitantly expensive.  

\subsection{Surrogate Models}

A surrogate model replaces the large cost of the model solve 
with a fast, explicit function evaluation.  The surrogate model
is constructed via interpolation or regression on a modest number 
of potentially deterministic traning samples $M$.  The
error from insufficient samples in the kernel density estimation is exchanged for the error between the true QoI $Q(\theta)$ and surrogate QoI $Q_S(\theta)$\cite{butler2013propagation}.
Sparse grid approaches \cite{bungartz2004,Jakeman2011LocalAD} give roughly the same error(modulo a factor of $n^{d-1}$, where $d$ is the dimension of the (input) parameter space and $n$ is the number of
training points)) as traditional tensor-product surrogates.
However, the number of samples is  $\mathcal{O}(2^n n^{d-1})$, instead of the full grid cost of $\mathcal{O}(2^{nd})$.  
These savings and accuracy can potentailly be increased by adopting adaptive sparse grid surrogates\cite{bungartz2004}.

In predictive extrapolation, the target scenario for $Q(\theta)$ is often sufficiently expensive to even make the surrogate 
approach tenuous.   One strategy in this
situation is to construct the surrogate using a sequence of lower-fidelity models to construct $\tilde{Q}(\theta)$, and then train the surrogate\cite{Pehermulti}.  This often requires
a good characterization of the error between $Q(\theta)$ and $\tilde{Q}(\theta)$.  This is not a well-explored or understood area in the kinetic plasma PIC community.  

Another approach is to add deterministic parameters that describes ``nearby'' scenarios,
s.t. $Q(\theta)=A(\theta,k_1,k_2,\ldots,k_r)$, $k_i$ fixed.  We call the determanistic parameters $k_i$ augmentation parameters, and the surrogate model
$A_S$ of $A$, the augmented surrogate.  Moderate gains are achieved when the number of augmentation parameters is small and the cost of sampling $A$ outside of the prediction
scenario $Q$ is much cheaper.  If the gradient of $A$ with respect to $\theta$ is only weakly dependent on the augmentation parameters, 
significant (cost-based) accuracy savings can be achieved.  

There are two fundmental assumptions in the construction of an augmented surrogate of
a $n$-dimensional predictive scenario $Q(\theta_1,\theta_2,\ldots,\theta_n)$.
\begin{enumerate}
\item There exists small number $m$ of (usually deterministic) parameters $d_1,d_2,\ldots,d_m$ that characterize
nearby $n$-dimensional predictive scenarios $\tilde{Q}_i(\theta_1,\theta_2,\ldots,\theta_n)$.
\item The cost of a sample from the nearby scenario, $C_{\tilde{Q}_i}$, is much less than the cost $C_Q$ of 
a sample from the desired prediction scenario.
\end{enumerate} 

The {\em augmented surrogate} is a surrogate model $A(\theta,d)$ constructed on training data $\{(\theta,d),\hat{Q}(\theta,d)\}$ in the $m+n$-dimensional
parameter space.   %We briefly illustrate this approach via the following numerical cartoon.

\subsubsection*{Adaptive sparse grid method}  
The classical sparse grid method is dimension agnostic\cite{bungartz2004}. All interactions
of the same order are treated equally. Often, a small subset of variables
and interactions contributes significantly to the variability of the function $Q(\theta,d)$.  If the variability in the $\theta$-dimensions is
greater than the variability in the scenario parameters($\{d_i\}$) then the overall cost of constructing the 
larger dimensional surrogate is actually less, due to the cheaper computational cost of $\tilde{Q}_i$.

We modify the 
greedy algorithm for constructing $h$-adaptive generalized sparse grid (h-GSG) in \cite{Jakeman2011LocalAD}.
The hierarchical surpluses for the current sparse grid are modified with a cost weight $W_C(d_1,d_2,\ldots,d_m)$ that approximates the relative
cost of simulating the added sparse grid point, and a $m$-dimensional distance metric that penalizes sparse grid
samples that are too far away from the prediction scenario $Q(\theta)$.  This encourages parameter exploration in $\theta_i$-dimensions at
inexpensive simulation levels, while rewarding coarse grid points that reduce the local surrogate error near the prediction scenario $Q(\theta)$.  


 


 


<<<<<<< HEAD
%
%\section{Motivation for Performing Study}
%>>>>>>> 6f21f4ca132eabe95ad55dea398ef4f20d0aadf1

The {\em augmented surrogate} is a surrogate model $A(\theta,d)$ constructed on training data $\{(\theta,d),\hat{Q}(\theta,d)\}$ in the $m+n$-dimensional
parameter space.   %We briefly illustrate this approach via the following numerical cartoon.

\subsubsection*{Adaptive sparse grid method}  
The classical sparse grid method is dimension agnostic \cite{Bungartz}. All interactions
of the same order are treated equally. Often, a small subset of variables
and interactions contributes significantly to the variability of the function $Q(\theta,d)$.  If the variability in the $\theta$-dimensions is
greater than the variability in the scenario parameters($\{d_i\}$) then the overall cost of constructing the 
larger dimensional surrogate is actually less, due to the cheaper computational cost of $\tilde{Q}_i$.

We modify the 
greedy algorithm for constructing $h$-adaptive generalized sparse grid (h-GSG) in \cite{Jakeman2011LocalAD}.
The hierarchical surpluses for the current sparse grid are modified with a cost weight $W_C(d_1,d_2,\ldots,d_m)$ that approximates the relative
cost of simulating the added sparse grid point, and a $m$-dimensional distance metric that penalizes sparse grid
samples that are too far away from the prediction scenario $Q(\theta)$.  This encourages parameter exploration in $\theta_i$-dimensions at
inexpensive simulation levels, while rewarding coarse grid points that reduce the local surrogate error near the prediction scenario $Q(\theta)$.  

\subsection{Proposed UQ Study}
We will conduct a base parameter scan in $(\rho_*)$, at the mean values of the uncertain inputs.  This will verify that wedge, Eulerian versus PIC, or other factors do not impact the conclusions reached in \cite{Yas_Ido}.  This scan will also provide the first training runs for
the augmented surrogate $A$.  

Data from a range of $\rho^{-1}_*$ simulations will be obtained in the interval (100,600) will be provided by Varis Carey and other members
of the BPS Scidac team  from simulations run at NERSC and Oak Ridge facilites.  A budget of 20,000 SU at NERSC has been allocated for providing the data for this scaling study(which is already partially complete at the time of this proposal)
 and for additional simulations to train the augmented surrogate to cover the input parameter space.
The main input parameters parameterize the location, slope, and shape of the core heat source and boundary sink.  Additional parameters covering
torque will be allowed if the computational budget allows further investigation.  If NERSC resources are expended, certain (small-scale)
simulations will be conducted on CU-Denver or RMCC resources(Summit) under the supervision of Varis Carey.

\subsubsection*{Postprocessing Software}
Postprocessing software (python and Matlab) are available to extract plasma QoI from  XGC 2D and 1D diagnostic output.  The BPS team will
arrange a NERSC account to avoid transfer of large binary files and allow simulation data to be repurposed for future studies.
\section{Educational Methodology}
In order to familiarize myself with the above experimental methods, I propose the following framework constructed of seeting up a toy problem, and implementing. First numerically evaluating a familiar PDE model using a finite difference method-  possibly the 1-D diffusion equation or the 1-D neutron transport equation - then propogating the uncertainty of the parameters through the model to develop a distribtuion of the quantity of interest. The neutron diffusion model is appealing as there are libraries available on the distribution of the nuclear cross-section that can be used for uncertainty quantification \cite{Smith}.\\
%%
After creating a blackbox to generate the quantity of interest under uncertainty, I can then use these generated quantities along with the avilable inputs to construct a surrogate model using methods of linear or nonlinear regression. The quadratic response surface model has been identified as a surrogate modeling method to perform this task. After constructing said surrogate model, the approximate quantity of interest can be compared to the quantity of interest of the real model, with statistics and correlations run on the QoI's. The surrogate models will be run under uncertainty as well, with a statistical comparison run on QoI distribution.To achieve a more accurate surrogate model, I will update the surrogate model based on the QoI distribution.\\
%%
The understand the complexity associated with scaling into N-D, as we will in this project, the finite difference model will progressively be scaled from 1-D to N-D. The above framework for uncertainty quantification and surrogate model creation and evaluation will be performed as the dimensionality of the system is increased. My thesis will contain all of this information as a guiding example for the actual research being performed.\\

% Go through details on push forward scanning using a surrogate model.

%The relationship between ion diffusivity, $\chi_i$, and the dimensionless radius $\rho*$ is given by
%$$
%\chi_i = (cT_e/|e|B )\rho*^{x_p}F(v_*, \beta, q_{\phi}, T_e/T_i, ...).
%$$
%
%The first coefficient is the Bohm diffusivity function, where $c$ is the speed of light in the plasma, $T_e$ is the electron temperature $e$ is the electron charge, and B is the perpendicular magnetic field. The second coefficient contains the dimensionless radius, where the exponent $x_p$ determines the scaling of the ion diffusivity with respect to the dimensionless radius. The third coefficient is the dimensinless group formed by all of the relevant parameters.\\
%\subsection{Methods of Reduced Order Models to Characterize Ion Heat Diffusion}

%
%  During this research project I will be focusing on the ion heat diffusivity, $\chi_i$, in a Tokamak reactor through the ion temperature gradient (ITG). I will be studying  scaling properties of the ITG with respect to the dimensionless number  $\rho* = \frac{\rho_i}{a}$, with $\rho* << 1$, and where $\rho_i$ is the ion gyroradius, or Larmor radius, and $a$ is the minor radius of a Tokamak reactor.\\


%Most research on the scaling of ion heat diffusivity is concerned with determining the exponential relationship between,  $\chi_i$ and $\chi_B\rho*^{n}$. $\chi_B = $ Bohm diffusivity function if the   on  I am interested in studying the scaling behavior of the ion diffusivity with respect to $\rho$, as it has been established that the scaling can be Bohm like, or $\chi_i \propto  \rho*^0$, or  The scaling properties of the ion diffusivity in a n, as if the diffusion scales in a Bohm (linear) fashion with respect to $\rho_*$ then the chan

%he  is definedwhich is evaluated by studying the ion temperature gradient (ITG). It has been established that the ion heat diffusivity scales with respect to variables such as the plasma temperature and the cross-sectional radius of the reactor core. If the scaling of the ion diffusivity is  In a fusion reactor we are interested in the scaling properties of the ion diffusivity with respect to the dimensionless parameter $\rho* = \frac{\rho_i}{a}$, where $rho_i$ is the ionic gyroradius, and $a$ is the minor radius. If the scaling of the ion heat difussivity It is possible for a power discrepancy of this nature to occur if the thermal conduction, or ion diffusivity, scales linearly with the temperature of the It has been shown that thermal conduction of ions out of the plasma have the potential scale in a Bohm fashion with respect to the minor radius of a Tokamak reactor.

%\section{Motivation}
%
%The motivation for this project is tied in to the overall goals of the Partnership Center for High-Fidelity Boundary Plasma Simulation, which is working to understand the boundary physics of a magnetically confined plasma in a nuclear fusion reactor using high-fidelity simulations. \cite{PPPL_P:2}\\
%
%For clarity, the boundary region in a fusion reactor is defined as “extending 10\% of the outer-minor radius in from the magnetic separatix, through the open field line scrape off layer, out to the material walls.” The separatix is the point where the magnetic field lines cross, which in the case of the Tokamak is at the bottom of of the toroid, while the scrape off layer (SOL) is defined as the plasma region that is characterized by open field lines, and is outside of the separatrix. The SOL absorbs most of the plasma exhaust and transports it along field lines to the divertor plates. The divertor plates are responsible for absorbing heat and ashe produced by the plasma, minimizing contamination of the plasma, and protecting thermal and neutronic loads.\\
%
%CHECK CITATION – INSERT MAIN MAGNETIC FIELD LINE FIGURE HERE\\
%
%The stability in the plasma boundary is critical to Tokamak operation, and thus the physics in the plasma boundary region must be understood before a fully functional fusion reactor can built. To elucidate the importance of understanding plasma boundary physics, an example of a critical issue related to stable operation of a fusion reactor is outlined below.\\
%
%Once a magnetically confined plasma reaches a heating threshold value the plasma transitions from a low-confinement mode (L-Mode) to a high-confinement mode (H-Mode). After L-H transition occurs, a steep pedestal in the plasma density develops in the plasma boundary region, as can be seen in figure 2.2. This transition brings a reduction in the radially directed electric field, as well as a reduction in the turbulence intensity, which in turn reduces heat transport, This reduction in turbulent transport leads to an increased heating in the ion core of the plasma by "a factor that is proportional to the temperature at the top of the pedestal." \cite{PPPL_P:2} The increased heating leads to a 2-3 fold increase in plasma power production, making the H-mode is the desired operating mode for future fusion reactors.\\
%
%Operating in H-mode requires a stable pedestal. However, the steep density gradient “acts as a source of free energy for the magnetohydrodynamic plasma edge localized modes (ELM),” \cite{PPPL_P:2}  in which the pedestal repeatedly “crashes”, yielding bursts of plasma towards the divertor plates. A proposed solution to this problem is to use stochastic magnetic fields to stabilize steep gradient in the boundary region, and thus control the edge localized modes.\\
%
%The Partnership seeks to understand  the L-H transition, pedestal structure, and  the requirements for ELM stability and control.  The plasma behavior in the boundary region is non-Maxwellian, and has non-equillibrium characteristics, requiring a first principles, 5-D gyrokinetic model, that simulates multiscale edge Tokamak plasma physics. Simulations include: The code used  (XGC), which is a particle in cell (PIC) code, requiring extreme high performance computing (HPC) to run a full plasma simulation.
%
%One of the Modeling and simulating the magnetically confined plasma in the boundary region is incredibly imporant to progressing plasma research due to the difficulty of collecting data from inside a nuclear reactor.\cite{Smith_UQ:3} The complexity of the  XGC code requires computational resources of the scale of Titan Cray XK at Oak Ridge National Laboratory. To reduce the computational complexity  to utilize surrogate methods to reduce the complexity of the models in order to more efficiently analyze the scaling of the ion diffusivity of the plasma. Constructing a surrogate model allows us to captures the primary behavior of the modeled process, and is sufficiently efficient for model validation and uncertainty propagation.  \cite{Smith_UQ:3}. To determine input parameters, boundary conditions and initial conditions, data from the C-Mod fusion reactor are analyzed in a probabilistic framework, in a process called model calibration. 
%
% \cite{PPPL_P:2}
%The boundary physics of a magnetically confined plasma within a reactor are tied to variety of parameters. The parameter of interest for this study is the ion diffusivity 
%As such, a bird's eye view of the current research being performed in the field of plasma physics will be presented. \cite{J_Friedberg:1}
%This will be followed by a an overview of the diffusion model that will be explored in this paper.\\

\section{Literature Review for MIS Thesis Proposal}


Determining the plasma size scaling of the ion diffusivity, and performing efficient sensitivity analysis on the ion diffusivity in the heating component of the XCG model, within a constructed surrogate model, will require a knowledge base in the following subjects.\\  

Ralph C. Smith has written an introductory textbook on uncertainty quantification that provides useful models, examples and problems to guide the educational and experimental components of this thesis. The texbook has all of the background knowledge necessary to develop a toy surrogate model as proposed in this document.\\  

\subsection*{Physics}

A background in plasma physics, and the component of the XCG model that is used to model plasma heating and ion diffusivity. To support my physical understanding while completing this research I have identified the following references.\\
\begin{description} %\begin{separation} must be directly above first item for some reason

\item[Plasma Physics and Fusion Reactors]\hfill

Jeffrey P. Friedberg – Plasma Physics and Fusion Energy\\
This textbook covers the physics of plasma fusion, its applications as an energy sources, the physical requirements to create an energy producing fusion reactor, and the designs requirements for fusion reactor with a toroidal geometry - the geometry of the ITER fusion reactor.\\

\item[Gyrokinetic Plasma Simulation]\hfill

Plasma Modeling Methods and Applications - Textbook\\
This is a modern textbook that covers kinetic theory plasma models, fluid equations and hybrid plasma models, and applications of these models. Each model is developed from first principles, making this an invaluable source for understanding the 5-D gyrokinetic model, as well as a good source for the developing and implementing the models that were discussed in the educational section of this proposal.\\

Dr. Wei-Lee from the Princeton Plasma Fusion Physics Laboratory has posted the lecture notes and homework assignments from a course on “Theory and Modeling of Kinetic Plasmas” on his website.\\
http://w3.pppl.gov/~wwlee/\\
This course contains the background information on gyrokinetic model that is being used to describe the plasma boundary physics in the ITER Tokamak reactor. \\
Project Description\\
The project proposal from the Partnership Center for High-Fidelity Boundary Plasma Simulation provides the motivation for performing this research, as well as a reference list containing relevant literature that will be reviewed and cited as necessary. \\

<<<<<<< HEAD
\subsection*{Numerical Methodology}
The mathematical component of my Master's of Integrated science will be fulfilled in this thesis through applications of numerical methods in accurate PDE model simulation, uncertainty quantification, surrogate model construction, and exact model validation.
\item[Uncertainty Quantification] \hfill

Ralph C. Smith - Uncertainty Quantification\\
Ralph C. Smith has written an introductory textbook on uncertainty quantification that provides useful models, examples and problems to guide the educational and experimental components of this thesis. The texbook has all of the background knowledge necessary to develop a toy surrogate model as proposed in this document.\\  

\item[Sparse Grids]
\item[Machine Learning]
\item[Case Study to Understand Current Methods in Uncertainty Quantification]\hfill

I am currently reviewing a paper referenced from the project proposal titled “Improved profile fitting and quantification of uncertainty in experimental measurements of impurity transport coefficients using Gaussian process regression” by Chilenski et al. to develop an understanding of the uncertainty quantification and parameter estimation pipeline.

\item[General Probability and Statistics]\hfill

Performing all of this work requires a base understanding in probability, Bayesian statistics, and data reduction and error analysis. The following textbooks have been identified to provide a base level of support in these topics.\\
=======
\item[Case Study to Understand Current Methods in Uncertainty Quantification]

I am currently reviewing a paper referenced from the project proposal titled “Improved profile fitting and quantification of uncertainty in experimental measurements of impurity transport coefficients using Gaussian process regression” by Chilenski et al. to develop an understanding of the uncertainty quantification and parameter estimation pipeline.

\item[Uncertainty Quantification]
Performing sensitivity analysis, and constructing surrogate models 
will require background knowledge in statistics, %both Bayesian and frequ error analysis, 
uncertainty quantification, and surrogate models. Two textbooks have been identified to support this work.\\
>>>>>>> 51d57c3de22a2c3682c977e27f2fbf388717be1f
\begin{enumerate}
\item Uncertainty Quantification: Theory, Implementation, and Applications by Ralph C. Smith\\
\item Data Reduction and Error Analysis for the Physical Sciences by Phillip R. Bevington and D. Keith Robinson\\
\end{enumerate}
\end{description}

%\section{Literature Review for MIS Thesis Proposal}
%\subsection{Numerical Methods}
%\begin{description}
%	\item[Uncertainty Quantification \& Surrogate Models]

%\
%\end{description}
%\subsection{Physics}
%\begin{description}
%An understanding of general plasma physics, tokamak reactor design, and plasma behavior withing a tokamak reactor, is required to complete this project. Understanding plasma behavior inside a tokamak reactor means understanding the 5-D gyrokinetic model for plasma behavior.  I have identified the following references to support my learning of the aforementioned physics.\\
%Jeffrey P. Friedberg – Plasma Physics and Fusion Energy\\
%This textbook covers the physics of plasma fusion, its applications as an energy sources, the physical requirements to create an energy producing fusion reactor, and the designs requirements for fusion reactor with a toroidal geometry - the geometry of the ITER fusion reactor.\\
%Plasma Modeling Methods and Applications\\
%This is a modern textbook that covers kinetic theory plasma models, fluid equations and hybrid plasma models, and applications of these models. Each model is developed from first principles, making this an invaluable source for understanding the 5-D gyrokinetic model, as well as a good source for the developing and implementing the models that were discussed in the educational section of this proposal
%Dr. Wei-Li Gyrokinetic Plasma Simulation\\
%Dr. Wei-Li from the Princeton Plasma Fusion Physics Laboratory has posted the lecture notes and homework assignments from a course on “Theory and Modeling of Kinetic Plasmas” on his website.\\
%http://w3.pppl.gov/~wwlee/.\\
%This course contains the background information on the gyrokinetic model that is being used to describe the plasma boundary physics in the ITER Tokamak reactor.\\
%Project Description\\
%The project proposal from the Partnership Center for High-Fidelity Boundary Plasma Simulation provides the motivation for performing this research, as well as a reference list containing relevant literature that will be reviewed and cited as necessary.\\
%\begin{enumerate}
%	\item[Uncertainty Quantification: Theory, Implementation, and Applications by Ralph C. Smith]
%	\item[Data Reduction and Error Analysis for the Physical Sciences by Phillip R. Bevington and D. Keith Robinson]
%\end{enumerate}
%\end{description}

\newpage
%% Bibliography Start
\bibliographystyle{ieeetr}
\bibliography{Bibli}
\end{document}
%%
%@book{J_Friedberg:1,
%	address = {New York},
%	edition = {1st},
%	title = {Plasma {Physics} and {Fusion} {Energy}},
%	isbn = {978-0-511-27375-9},
%	language = {English},
%	publisher = {Cambridge University Press},
%	author = {{Jeffrey P. Friedberg}},
%	year = {2007}
%}
%
%@book{carl_edward_rasmussen_gaussian_2006,
%	title = {Gaussian {Processes} for {Machine} {Learning}},
%	isbn = {0-262-18253-X},
%	language = {English},
%	publisher = {MIT Press},
%	author = {{Carl Edward Rasmussen} and {Christopher K.I. Williams}},
%	year = {2006},
%	file = {Gaussian Processes for Machine Learning.pdf:C\:\\Users\\EvanD\\OneDrive\\Documents\\MIS Thesis\\Papers and Textbooks\\Gaussian Processes for Machine Learning.pdf:application/pdf}
%}
%
%@misc{PPPL_P:2,
%	title = {Partnership {Center} for {High}-{Fidelity} {Boundary} {Plasma} {Simulation} {Project} {Proposal}},
%	language = {English},
%	publisher = {Princeton},
%	author = {{Choong-Seock Chang et al}},
%	year = {2017}
%}
%
%
%
%@article{m.a._chilenski_et_al_improved_2015,
%	title = {Improved profile fitting and quantification of uncertainty in experimental measurements of impurity transport coefficients using {Gaussian} process regression},
%	volume = {55},
%	number = {023012},
%	journal = {Nucl. Fusion},
%	author = {{M.A. Chilenski et al}},
%	year = {2015},
%	pages = {21},
%	file = {UQ_Impurity_Transport_Coefficients_GCR_Chilenski.pdf:C\:\\Users\\EvanD\\OneDrive\\Documents\\MIS Thesis\\Papers and Textbooks\\UQ_Impurity_Transport_Coefficients_GCR_Chilenski.pdf:application/pdf}
%}
%
%@techreport{benjamin_peherstorfer_survey_2016,
%	title = {{SURVEY} {OF} {MULTIFIDELITY} {METHODS} {IN} {UNCERTAINTY} {PROPAGATION}, {INFERENCE}, {AND} {OPTIMIZATION}},
%	url = {http://web.mit.edu/pehersto/www/preprints/multi-fidelity-survey-peherstorfer-willcox-gunzburger.pdf},
%	institution = {MIT.edu},
%	author = {{Benjamin Peherstorfer} and {Karen Willcox} and {Max Gunzburger}},
%	month = jun,
%	year = {2016},
%	pages = {57},
%	file = {multi-fidelity-survey-peherstorfer-willcox-gunzburger.pdf:C\:\\Users\\EvanD\\OneDrive\\Documents\\MIS Thesis\\Papers and Textbooks\\multi-fidelity-survey-peherstorfer-willcox-gunzburger.pdf:application/pdf}
%}
%@book{Smith_UQ:3,
%	title ={Uncertinaty Quantification - Theory, Implementation, and Applications},
%	isbn = {978-1-611973-21-1},
%	language = {English},
%	publisher = {Society of Industrial and Applied Mathematics},
%	author = {Ralph C. Smith},
%	year = {2014}
%}

