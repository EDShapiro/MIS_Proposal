\documentclass{article}
\usepackage{amsmath}
\usepackage{amssymb}
\usepackage{gensymb}
\usepackage{hyperref}
\usepackage{listings}
\usepackage{graphicx}
\usepackage[margin=1.25in]{geometry}
\linespread{2.0}
\graphicspath{ {c:/Users/EvanD/OneDrive/Pictures/}}
\title{Integrated Science Thesis Proposal}
\author{Evan Shapiro \\ Master's of Integrated Science, University of Colorado Denver}
\begin{document}


\begin{abstract}
Plasma microturbulence is a leading candidate of the source
%for the anomalous
of anomalous diffusion observed in %modern 
tokamak experiments. Global gyrokinetic
models have been shown to accurately predict the ion diffusivity, $\chi_i$
. Extreme-scale, fixed-flux supercomputing simulations are beginning
to simulate modes of operation relevant to next-generation(ITER,DEMO)
reactors. Using a surrogate model to reduce the computational expense of XGC simulations, I conduct a predictive scan
in $\rho^{-1}$ , to ascertain
whether or not the ion diffusivity $\chi_i$ scales in a Bohm or gyro-Bohm
fashion, and analyze the sensitivity of $\chi_i$ to perturbation in the heating
model.% in the $XGC$ model.
\end{abstract}

\maketitle
\tableofcontents

\section{Introduction}
\subsection{Fusion Energy as a Potential Source of Electricity}
One of the most pressing issues of the 21st century is providing affordable, sustainable, safe energy to the growing world population.  As of 2014, there were still 1.1 billion people without access to any form of electricity, with an estimated 3 billion people without access to carbon free energy. Anthropogenically driven climate change has constrained this problem, making policy makers rethink long terms policies regarding coal and natural gas, while safety issues, and public perception, has led to a decreased investment in nuclear fission reactors in the US.\\
Renewable, clean sources of energy, like large scale solar farms and wind turbine farms, are being adopted in many regions trying to adapt to the the energy pressures presented by climate change. In Colorado, Xcel Energy has projected that the solar energy capacity will reach 342 MW by 2019, with 55 percent of all electricity being generated by a mixture of renewable energy sources by 2026. Germany's renewable energy capacity is capable of prividing 95 percent of electrical demands at peak out, and expects that renewables will provide 18 percent of all power needs by 2020.\\

It is tempting to extrapolate on this trend and expect that solar and wind power will eventually replace coal or nuclear fission power plants, however, due to the undependable nature of the sun and the wind, providing continuous electricity requires a base load power supply which has historically been supplied by nuclear fission and coal power plants. Due to the above mentioned issues, it is desirable to replace the current baseloads with an energy source that is both safe and clean, and  magnetic nuclear fusion has the potential to be this replacement energy source.\\
Nuclear fusion occurs when two light elements, such as tritium and deuterium, collide with enough energy for a component of one of the nucleii, such as a proton in the case of a deuterium-tritium collision, to be exchanged  between the nucleii.The dueterium-tritium collision is presented below:
$$
D+T \rightarrow \alpha + n + 17.6 MeV
$$  
\cite{J_Friedberg:1} .This reaction describes the fusing of the deuterium proton to the tritium nucleus, resulting in the release of an $\alpha$ particle, a neutron, and $17.6 MeV$ of the nuclear binding energy, with $14.1 MeV$ being transferred to the neutron particle nd $3.5 MeV$ transferred to the $\alpha$ particle as kinetic energy. It is this reaction, and tapping into the subsequent release of energy , that drives current fusion research. Creating a sustained reaction of a 50-50 $\%$ Deuterium-Tritium mixture requires an input of $70KeV$, corresponding to a temperature on the order of $100*10^6$K. It is this high temperature environment that leads to the complete ionization of the deuterium-tritium mixture, resulting in a plasma.  \\

More detail will be included below - just need to get ideas down. 

Due to requirements for magnetic pressure, plasma confinement time, and plasma temperature, with the complete explanation being beyond the scope of this proposal, a toroidal geometry is the accepted geometry for a fusion reactor. The basic design requirements are as follows. The plasma is contained within the inner cylinder of a toroid, with the first wall of the cylinder being protected by a strong magnetic field that shields the wall from thermal loading from the $14.1 MeV$ neutrons,  Bremsstrahlung heat loss, and heat conduction. Inside the first wall is the region called the blanket and shield, which is the region of heat exchange for powering an external steam turbine, and provides a tritium breeding ground that sustains the supply of tritium required for the fusion reaction. Outside of this region is the a shield wall which prevents the escape of any radioactive neutrons or gamma particles. The final layer of the toroid contains the superconducting toroidal solenoid, which generates the $10-15 [T]$ magnetic field inside of the fusion reactor. Two schematics can be found below - one of the theoretical cross-sectinal area of toroidal reactor, and one of the ARIES-AT tokamak  design.

Basic schematic of fusion reactor with heat exchanger here.

It should be noted that the safety of fusion reactors is related to fusion core containing only enough fuel to maintain the plasma, whereas fission reactors have historically contained enough radiative material to power a plant for years, leading to the possibility of meltdown if cooling systems fail.
\subsection{Meeting Power Balance in a Fusion Reactor}
Before delving into the issue that will be addressed in the proposed research project, one must have a firm understanding of the conditions required in a fusion reactor to maintain a fusion reaction.	To get a basic understanding of the power requirements necessary for maintaining a stable plasma, we can model the energy content and power production of a plasma reactor using the fluid dynamics conservation of energy equation
$$
\frac{3}{2}\frac{\partial p}{\partial t} + \frac{3}{2}\nabla \cdot p \vec{v} + p\nabla \cdot v + \nabla\cdot \vec{q} = S
$$
where I have substituted the internal energy $U$ for $\frac{3}{2}p$ without showing the full derivation, which can be found in J. Friedberg's text on plasma fusion \cite{J_Friedberg:1}.  \\

This equation accounts for all of the physical processes that contribute to or take energy out of the volume of plasma. The first term account for the variation of energy flowing into and out of the system with respect to time, the second term accounts for energy lost or gained due to convection of heat out of the system,  the third term accounts for energy losses due to compression and expansion of the fluid within a reactor, the 4th term account for heat losses due to diffusion, and $S$ accounts for the sources and sinks available to the plasma. If we assume a steady state then the the time term drops out, and no convection or compression is occuring, leaving the 0-D energy conservation equation:
$$
 \nabla\cdot \vec{q} = S.
$$
indicating that at steady-state a fusion reactor's diffusive processes must be matched by the net sources and sinks in the system for power balance to occur.%Is this the correct interpretation of this equation?\\
%Is this enough detail on power balance? Should I go into more detail on what the sources and sinks are?

\subsection{Characterizing Ion Heat Diffusion in a Tokamak Reactor}
\subsection{Methods of Reduced Order Models to Characterize Ion Heat Diffusion}
The energy and ion confinement properties of a magnetically confined plasma are critical to reaching a state of power balance in a fusion reactor reactor. If power balance is not met then the alpha particle power heating from the plasma fusion reactions is not enough to overcome the losses due to Bremmstrahlung and thermal conduction by ions and electrons \cite{J_Friedberg:1}. It is known that the ion diffusivity in Tokamak reactors stellarators has a time scale that can not be explained solely by Coulomb interactions \cite{Gyro_Petty}.  It has been postulated that plasma turbulence is responsible for the non-Coulombic thermal conductivity, however due to the complexity of modeling turbulence other methods have been used to understand how the ion temperature gradient scales with reactor size.\\

Prior research \cite{Yas_Ido}  has focused on the scaling relationship between the ion thermal diffusivity, $\chi_i$, and the dimensionless variable  $\frac{\rho_i}{a}$,  where $\rho_i$ is the ion gyroradius, or Larmor radius, and $a$ is the minor radius of a Tokamak reactor;  it can be seen that  $\frac{\rho_i}{a}<<1$.\\

Of interest is whether the ion diffusivity scaling is Bohm-like, or scales linearly with temperature, or gyro-Bohm like, scaling sublinearly with temperature.\\

Previous studies on the relationship between ion diffusivity and reactor size have been completed in a reduced scale geometry \cite{Yas_Ido}, leaving

% Go through details on push forward scanning using a surrogate model.

The relationship between ion diffusivity, $\chi_i$, and the dimensionless radius $\rho*$ is given by
$$
\chi_i = (cT_e/|e|B )\rho*^{x_p}F(v_*, \beta, q_{\phi}, T_e/T_i, ...).
$$

The first coefficient is the Bohm diffusitivity function, where $c$ is the speed of light in the plasma, $T_e$ is the electron temperature $e$ is the electron charge, and B is the perpendicular magnetic field. The second coefficient contains the dimensionless radius, where the exponent $x_p$ determines the scaling of the ion diffusivity with respect to the dimensionless radius. The third coefficient is the dimensinless group formed by all of the relevant parameters.\\

In this thesis I want to determine the scaling behavior of $\chi_i$ with respect to $\rho_*$ by performing a scan in $\rho_*$ using a surrogate model. The reason for using a surrogate model is that the full-f gyrokinetic model that is used to model the behavior of the plasma is computationally expensive, requiring extreme scale HPC power, like the Titan supercomputer at Oak Ridge National Laboratories.
%
%  During this research project I will be focusing on the ion heat diffusivity, $\chi_i$, in a Tokamak reactor through the ion temperature gradient (ITG). I will be studying  scaling properties of the ITG with respect to the dimensionless number  $\rho* = \frac{\rho_i}{a}$, with $\rho* << 1$, and where $\rho_i$ is the ion gyroradius, or Larmor radius, and $a$ is the minor radius of a Tokamak reactor.\\


Most research on the scaling of ion heat diffusivity is concerned with determining the exponential relationship between,  $\chi_i$ and $\chi_B\rho*^{n}$. $\chi_B = $ Bohm diffusivity function if the   on  I am interested in studying the scaling behavior of the ion diffusivity with respect to $\rho$, as it has been established that the scaling can be Bohm like, or $\chi_i \propto  \rho*^0$, or  The scaling properties of the ion diffusivity in a n, as if the diffusion scales in a Bohm (linear) fashion with respect to $\rho_*$ then the chan

%he  is definedwhich is evaluated by studying the ion temperature gradient (ITG). It has been established that the ion heat diffusivity scales with respect to variables such as the plasma temperature and the cross-sectional radius of the reactor core. If the scaling of the ion diffusivity is  In a fusion reactor we are interested in the scaling properties of the ion diffusivity with respect to the dimensionless parameter $\rho* = \frac{\rho_i}{a}$, where $rho_i$ is the ionic gyroradius, and $a$ is the minor radius. If the scaling of the ion heat difussivity It is possible for a power discrepancy of this nature to occur if the thermal conduction, or ion diffusivity, scales linearly with the temperature of the It has been shown that thermal conduction of ions out of the plasma have the potential scale in a Bohm fashion with respect to the minor radius of a Tokamak reactor.

\section{Prediction under Uncertainty}
%
% Overview of the Problem
%

Prediction under uncertainty is often an expensive and complicated process\cite{odensiam1,odensiam2}.
The tradeoff between the computational cost of complex models and the loss of predictive accuracy 
associated with simpler models may be addressed by emerging multifidelity UQ approaches\cite{ngmulti,Pehermulti}.

In magnetic confinement fusion, there is an important, clearly defined set of prediction scenarios, corresponding 
to modelling future reactor(ITER/DEMO) performance.  Even a ``high-fidelity'', extreme-scale model such as XGC, still
has a parameter space of large enough dimension to make a brute-force, 
sampling-based predictive process
impossible.  
%The $N^{-1/2}$ convergence, where $N$ is the number of samples of the quantity of interest(QOI), is infeasible given the requirements on computational resources. 
The traditional approach to prediction, with or without extrapolation, is to sample the model parameter input space
$\theta=\theta_1,\theta_2, \ldots, \theta_d$, evaluate the model, and return a quantity of interest (QoI) $Q(\theta)$.
Most realistic QoI maps are nonlinear in the QoI map(even if the governing PDE is linear), 
so the probability distribution function(PDF) of $Q(\theta)$ will
have to be estimate in a non-parametric way, typically
 by kernel density estimation, or by computing empirical 
statistics from samples.
The mean-integrated squared error
of the approximate PDF converges with a rate of $\mathcal{O}(N^{-2/(d+4)})$, where $N$ is the number of samples of the model
and $d$ is the dimension of the input space.  As each sample typically involves a PDE solve and subsequent post-processing,
this process quickly becomes exorbitantly expensive.  

\subsection{Surrogate Models}

A surrogate model replaces the large cost of the model solve 
with a fast, explicit function evaluation.  The surrogate model
is constructed via interpolation or regression on a modest number 
of potentially deterministic traning samples $M$.  The
error from insufficient samples in the kernel density estimation is exchanged for the error between the true QoI $Q(\theta)$ and surrogate QoI $Q_S(\theta)$\cite{butler2013propagation}.
Sparse grid approaches \cite{bungartz2004,Jakeman2011LocalAD} give roughly the same error(modulo a factor of $n^{d-1}$, where $d$ is the dimension of the (input) parameter space and $n$ is the number of
training points)) as traditional tensor-product surrogates.
However, the number of samples is  $\mathcal{O}(2^n n^{d-1})$, instead of the full grid cost of $\mathcal{O}(2^{nd})$.  
These savings and accuracy can potentailly be increased by adopting adaptive sparse grid surrogates\cite{bungartz2004}.

\subsubsection*{Augmented Surrogates}

In predictive extrapolation, the target scenario for $Q(\theta)$ is often sufficiently expensive to even make the surrogate 
approach tenuous.   One strategy in this
situation is to construct the surrogate using a sequence of lower-fidelity models to construct $\tilde{Q}(\theta)$, and then train the surrogate\cite{Pehermulti}.  This often requires
a good characterization of the error between $Q(\theta)$ and $\tilde{Q}(\theta)$.  This is not a well-explored or understood area in the kinetic plasma PIC community.  

Another approach is to add deterministic parameters that describes ``nearby'' scenarios,
s.t. $Q(\theta)=A(\theta,k_1,k_2,\ldots,k_r)$, $k_i$ fixed.  We call the determanistic parameters $k_i$ augmentation parameters, and the surrogate model
$A_S$ of $A$, the augmented surrogate.  Moderate gains are achieved when the number of augmentation parameters is small and the cost of sampling $A$ outside of the prediction
scenario $Q$ is much cheaper.  If the gradient of $A$ with respect to $\theta$ is only weakly dependent on the augmentation parameters, 
significant (cost-based) accuracy savings can be achieved.  

There are two fundmental assumptions in the construction of an augmented surrogate of
a $n$-dimensional predictive scenario $Q(\theta_1,\theta_2,\ldots,\theta_n)$.
\begin{enumerate}
\item There exists small number $m$ of (usually deterministic) parameters $d_1,d_2,\ldots,d_m$ that characterize
nearby $n$-dimensional predictive scenarios $\tilde{Q}_i(\theta_1,\theta_2,\ldots,\theta_n)$.
\item The cost of a sample from the nearby scenario, $C_{\tilde{Q}_i}$, is much less than the cost $C_Q$ of 
a sample from the desired prediction scenario.
\end{enumerate} 

The {\em augmented surrogate} is a surrogate model $A(\theta,d)$ constructed on training data $\{(\theta,d),\hat{Q}(\theta,d)\}$ in the $m+n$-dimensional
parameter space.   %We briefly illustrate this approach via the following numerical cartoon.

\subsubsection*{Adaptive sparse grid method}  
The classical sparse grid method is dimension agnostic\cite{bungartz2004}. All interactions
of the same order are treated equally. Often, a small subset of variables
and interactions contributes significantly to the variability of the function $Q(\theta,d)$.  If the variability in the $\theta$-dimensions is
greater than the variability in the scenario parameters($\{d_i\}$) then the overall cost of constructing the 
larger dimensional surrogate is actually less, due to the cheaper computational cost of $\tilde{Q}_i$.

We modify the 
greedy algorithm for constructing $h$-adaptive generalized sparse grid (h-GSG) in \cite{Jakeman2011LocalAD}.
The hierarchical surpluses for the current sparse grid are modified with a cost weight $W_C(d_1,d_2,\ldots,d_m)$ that approximates the relative
cost of simulating the added sparse grid point, and a $m$-dimensional distance metric that penalizes sparse grid
samples that are too far away from the prediction scenario $Q(\theta)$.  This encourages parameter exploration in $\theta_i$-dimensions at
inexpensive simulation levels, while rewarding coarse grid points that reduce the local surrogate error near the prediction scenario $Q(\theta)$.  

\subsection{Proposed UQ Study}
We will conduct a base parameter scan in $(\rho_*)$, at the mean values of the uncertain inputs.  
This will verify that wedge, Eulerian versus PIC, or other factors do not impact the conclusions reached in \cite{Yas_Ido}.  This scan will also provide the first training runs for
the augmented surrogate $A$.  

Data from a range of $\rho^{-1}_*$ simulations in the interval (100,600) will be provided by Varis Carey and other members
of the HBPS Scidac team  from supercomputer simulations run at NERSC and Oak Ridge facilites.  A budget of 20,000 SU at NERSC has been allocated for providing the data for this scaling study(which is already partially complete at the time of this proposal)
 and for additional simulations to train the augmented surrogate to cover the input parameter space.
The main input parameters parameterize the location, slope, and shape of the core heat source and boundary sink.  Additional parameters covering
torque will be allowed if the computational budget allows further investigation.  If NERSC resources are expended, certain (small-scale)
simulations will be conducted on CU-Denver or RMCC resources(Summit) under the supervision of Varis Carey.

\subsubsection*{Postprocessing Software}
Postprocessing software (python and Matlab) are available to extract plasma QoI from  XGC 2D and 1D diagnostic output.  The BPS team will
arrange a NERSC account for Evan to avoid transfer of large binary files and allow simulation data to be repurposed for serving the 
plasma fusion community.

%Evan: you won't be expected to do any major software development(mainly just modify some scripts, maybe improve them) as part of this project.

\subsection{Contingency Research Plan}
Large scale gyrokinetic simulations, especially at large values of $\rho_*^{-1}$, involve considerable resources, and certain stability 
requirements for the simulations are only roughly known {\em a priori}.  In the event of insufficient data to build the $\rho_*$ augmented surrogate, we will instead switch to looking at the effect on transport of uncertanties in the background magnetic field.  This is actually
a harder problem as the perturbations (i.e. the samples) must be valid solutions of the Grad-Shrafanov equations.  We will investigate this in a 1D or 2D slab geometry using standard techniques uncertainty quantification techniques.  This will also be a publishable, impactful result, and could affect future work in 2D(axisymmetric) or 3D tokamak geometries.


 

 



\section{Motivation for Performing Study}

The motivation for this project is tied in to the overall goals of the Partnership Center for High-Fidelity Boundary Plasma Simulation, which is working to understand the boundary physics of a magnetically confined plasma in a nuclear fusion reactor using high-fidelity simulations. \cite{PPPL_P:2}\\

For clarity, the boundary region in a fusion reactor is defined as “extending 10\% of the outer-minor radius in from the magnetic separatix, through the open field line scrape off layer, out to the material walls.” The separatix is the point where the magnetic field lines cross, which in the case of the Tokamak is at the bottom of of the toroid, while the scrape off layer (SOL) is defined as the plasma region that is characterized by open field lines, and is outside of the separatrix. The SOL absorbs most of the plasma exhaust and transports it along field lines to the divertor plates. The divertor plates are responsible for absorbing heat and ashe produced by the plasma, minimizing contamination of the plasma, and protecting thermal and neutronic loads.\\

CHECK CITATION – INSERT MAIN MAGNETIC FIELD LINE FIGURE HERE\\

The stability in the plasma boundary is critical to Tokamak operation, and thus the physics in the plasma boundary region must be understood before a fully functional fusion reactor can built. To elucidate the importance of understanding plasma boundary physics, an example of a critical issue related to stable operation of a fusion reactor is outlined below.\\

Once a magnetically confined plasma reaches a heating threshold value the plasma transitions from a low-confinement mode (L-Mode) to a high-confinement mode (H-Mode). After L-H transition occurs, a steep pedestal in the plasma density develops in the plasma boundary region, as can be seen in figure 2.2. This transition brings a reduction in the radially directed electric field, as well as a reduction in the turbulence intensity, which in turn reduces heat transport, This reduction in turbulent transport leads to an increased heating in the ion core of the plasma by "a factor that is proportional to the temperature at the top of the pedestal." \cite{PPPL_P:2} The increased heating leads to a 2-3 fold increase in plasma power production, making the H-mode is the desired operating mode for future fusion reactors.\\

Operating in H-mode requires a stable pedestal. However, the steep density gradient “acts as a source of free energy for the magnetohydrodynamic plasma edge localized modes (ELM),” \cite{PPPL_P:2}  in which the pedestal repeatedly “crashes”, yielding bursts of plasma towards the divertor plates. A proposed solution to this problem is to use stochastic magnetic fields to stabilize steep gradient in the boundary region, and thus control the edge localized modes.\\

The Partnership seeks to understand  the L-H transition, pedestal structure, and  the requirements for ELM stability and control.  The plasma behavior in the boundary region is non-Maxwellian, and has non-equillibrium characteristics, requiring a first principles, 5-D gyrokinetic model, that simulates multiscale edge Tokamak plasma physics. Simulations include: The code used  (XGC), which is a particle in cell (PIC) code, requiring extreme high performance computing (HPC) to run a full plasma simulation.

One of the Modeling and simulating the magnetically confined plasma in the boundary region is incredibly imporant to progressing plasma research due to the difficulty of collecting data from inside a nuclear reactor.\cite{Smith_UQ:3} The complexity of the  XGC code requires computational resources of the scale of Titan Cray XK at Oak Ridge National Laboratory. To reduce the computational complexity  to utilize surrogate methods to reduce the complexity of the models in order to more efficiently analyze the scaling of the ion diffusivity of the plasma. Constructing a surrogate model allows us to captures the primary behavior of the modeled process, and is sufficiently efficient for model validation and uncertainty propagation.  \cite{Smith_UQ:3}. To determine input parameters, boundary conditions and initial conditions, data from the C-Mod fusion reactor are analyzed in a probabilistic framework, in a process called model calibration. 

 \cite{PPPL_P:2}
The boundary physics of a magnetically confined plasma within a reactor are tied to variety of parameters. The parameter of interest for this study is the ion diffusivity 
As such, a bird's eye view of the current research being performed in the field of plasma physics will be presented. \cite{J_Friedberg:1}
This will be followed by a an overview of the diffusion model that will be explored in this paper.\\
\subsubsection{Magnetically Confined Plasma as an Energy Source}
\subsubsection{Physics of Magnetically Confined Fusion}
\subsubsection{Transport Phenomenon in Fusion}
\section{Modeling Section}
\section{Literature Review}
Literature Review for MIS Thesis Proposal
Ralph C. Smith - Uncertainty Quantification\\
Determining the plasma size scaling of the ion diffusivity, and performing efficient sensitivity analysis on the ion diffusivity in the heating component of the XCG model, within a constructed surrogate model, will require a knowledge base in the following subjects.\\  
Physics\\
A background in plasma physics, and the component of the XCG model that is used to model plasma heating and ion diffusivity. To support my physical understanding while completing this research I have identified the following references.\\
Plasma Physics and Fusion Reactors\\
Jeffrey P. Friedberg – Plasma Physics and Fusion Energy\\
This textbook covers the physics of plasma fusion, its applications as an energy sources, the physical requirements to create an energy producing fusion reactor, and the designs requirements for fusion reactor with a toroidal geometry - the geometry of the ITER fusion reactor.\\
XCG Modeling\\
Dr. Wei-Lee from the Princeton Plasma Fusion Physics Laboratory has posted the lecture notes and homework assignments from a course on “Theory and Modeling of Kinetic Plasmas” on his website.\\
http://w3.pppl.gov/~wwlee/\\
 This course contains the background information on gyrokinetic model that is being used to describe the plasma boundary physics in the ITER Tokamak reactor. \\
Project Description\\
The project proposal from the Partnership Center for High-Fidelity Boundary Plasma Simulation provides the motivation for performing this research, as well as a reference list containing relevant literature that will be reviewed and cited as necessary. \\
Case Study to Understand Current Methods in Uncertainty Quantification\\
I am currently reviewing a paper referenced from the project proposal titled “Improved profile fitting and quantification of uncertainty in experimental measurements of impurity transport coefficients using Gaussian process regression” by Chilenski et al. to develop an understanding of the uncertainty quantification and parameter estimation pipeline.\\
Uncertainty Quantification\\
Performing the sensitivity analysis, and constructing surrogate models will require background knowledge in statistics, both Bayesian and frequ error analysis, uncertainty quantification, and surrogate models. Two textbooks have been identified to support this work.\\
Uncertainty Quantification: Theory, Implementation, and Applications by Ralph C. Smith\\
Data Reduction and Error Analysis for the Physical Sciences by Phillip R. Bevington and D. Keith Robinson\\

\newpage
%% Bibliography Start
\bibliographystyle{ieeetr}
\bibliography{Bibli}
\end{document}
%%
%@book{J_Friedberg:1,
%	address = {New York},
%	edition = {1st},
%	title = {Plasma {Physics} and {Fusion} {Energy}},
%	isbn = {978-0-511-27375-9},
%	language = {English},
%	publisher = {Cambridge University Press},
%	author = {{Jeffrey P. Friedberg}},
%	year = {2007}
%}
%
%@book{carl_edward_rasmussen_gaussian_2006,
%	title = {Gaussian {Processes} for {Machine} {Learning}},
%	isbn = {0-262-18253-X},
%	language = {English},
%	publisher = {MIT Press},
%	author = {{Carl Edward Rasmussen} and {Christopher K.I. Williams}},
%	year = {2006},
%	file = {Gaussian Processes for Machine Learning.pdf:C\:\\Users\\EvanD\\OneDrive\\Documents\\MIS Thesis\\Papers and Textbooks\\Gaussian Processes for Machine Learning.pdf:application/pdf}
%}
%
%@misc{PPPL_P:2,
%	title = {Partnership {Center} for {High}-{Fidelity} {Boundary} {Plasma} {Simulation} {Project} {Proposal}},
%	language = {English},
%	publisher = {Princeton},
%	author = {{Choong-Seock Chang et al}},
%	year = {2017}
%}
%
%
%
%@article{m.a._chilenski_et_al_improved_2015,
%	title = {Improved profile fitting and quantification of uncertainty in experimental measurements of impurity transport coefficients using {Gaussian} process regression},
%	volume = {55},
%	number = {023012},
%	journal = {Nucl. Fusion},
%	author = {{M.A. Chilenski et al}},
%	year = {2015},
%	pages = {21},
%	file = {UQ_Impurity_Transport_Coefficients_GCR_Chilenski.pdf:C\:\\Users\\EvanD\\OneDrive\\Documents\\MIS Thesis\\Papers and Textbooks\\UQ_Impurity_Transport_Coefficients_GCR_Chilenski.pdf:application/pdf}
%}
%
%@techreport{benjamin_peherstorfer_survey_2016,
%	title = {{SURVEY} {OF} {MULTIFIDELITY} {METHODS} {IN} {UNCERTAINTY} {PROPAGATION}, {INFERENCE}, {AND} {OPTIMIZATION}},
%	url = {http://web.mit.edu/pehersto/www/preprints/multi-fidelity-survey-peherstorfer-willcox-gunzburger.pdf},
%	institution = {MIT.edu},
%	author = {{Benjamin Peherstorfer} and {Karen Willcox} and {Max Gunzburger}},
%	month = jun,
%	year = {2016},
%	pages = {57},
%	file = {multi-fidelity-survey-peherstorfer-willcox-gunzburger.pdf:C\:\\Users\\EvanD\\OneDrive\\Documents\\MIS Thesis\\Papers and Textbooks\\multi-fidelity-survey-peherstorfer-willcox-gunzburger.pdf:application/pdf}
%}
%@book{Smith_UQ:3,
%	title ={Uncertinaty Quantification - Theory, Implementation, and Applications},
%	isbn = {978-1-611973-21-1},
%	language = {English},
%	publisher = {Society of Industrial and Applied Mathematics},
%	author = {Ralph C. Smith},
%	year = {2014}
%}
%%
