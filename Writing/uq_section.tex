%
% Overview of the Problem
%

Prediction under uncertainty is often an expensive and complicated process\cite{odensiam1,odensiam2}.
The tradeoff between the computational cost of complex models and the loss of predictive accuracy 
associated with simpler models may be addressed by emerging multifidelity UQ approaches\cite{ngmulti,Pehermulti}.

In magnetic confinement fusion, there is an important, clearly defined set of prediction scenarios, corresponding 
to modelling future reactor(ITER/DEMO) performance.  Even a ``high-fidelity'', extreme-scale model such as XGC, still
has a parameter space of large enough dimension to make a brute-force, 
sampling-based predictive process
impossible.  
%The $N^{-1/2}$ convergence, where $N$ is the number of samples of the quantity of interest(QOI), is infeasible given the requirements on computational resources. 
The traditional approach to prediction, with or without extrapolation, is to sample the model parameter input space
$\theta=\theta_1,\theta_2, \ldots, \theta_d$, evaluate the model, and return a quantity of interest (QoI) $Q(\theta)$.
Most realistic QoI maps are nonlinear in the QoI map(even if the governing PDE is linear), 
so the probability distribution function(PDF) of $Q(\theta)$ will
have to be estimate in a non-parametric way, typically
 by kernel density estimation, or by computing empirical 
statistics from samples.
The mean-integrated squared error
of the approximate PDF converges with a rate of $\mathcal{O}(N^{-2/(d+4)})$, where $N$ is the number of samples of the model
and $d$ is the dimension of the input space.  As each sample typically involves a PDE solve and subsequent post-processing,
this process quickly becomes exorbitantly expensive.  

\subsection{Surrogate Models}

A surrogate model replaces the large cost of the model solve 
with a fast, explicit function evaluation.  The surrogate model
is constructed via interpolation or regression on a modest number 
of potentially deterministic traning samples $M$.  The
error from insufficient samples in the kernel density estimation is exchanged for the error between the true QoI $Q(\theta)$ and surrogate QoI $Q_S(\theta)$\cite{butler2013propagation}.
Sparse grid approaches \cite{bungartz2004,Jakeman2011LocalAD} give roughly the same error(modulo a factor of $n^{d-1}$, where $d$ is the dimension of the (input) parameter space and $n$ is the number of
training points)) as traditional tensor-product surrogates.
However, the number of samples is  $\mathcal{O}(2^n n^{d-1})$, instead of the full grid cost of $\mathcal{O}(2^{nd})$.  
These savings and accuracy can potentailly be increased by adopting adaptive sparse grid surrogates\cite{bungartz2004}.

\subsubsection*{Augmented Surrogates}

In predictive extrapolation, the target scenario for $Q(\theta)$ is often sufficiently expensive to even make the surrogate 
approach tenuous.   One strategy in this
situation is to construct the surrogate using a sequence of lower-fidelity models to construct $\tilde{Q}(\theta)$, and then train the surrogate\cite{Pehermulti}.  This often requires
a good characterization of the error between $Q(\theta)$ and $\tilde{Q}(\theta)$.  This is not a well-explored or understood area in the kinetic plasma PIC community.  

Another approach is to add deterministic parameters that describes ``nearby'' scenarios,
s.t. $Q(\theta)=A(\theta,k_1,k_2,\ldots,k_r)$, $k_i$ fixed.  We call the determanistic parameters $k_i$ augmentation parameters, and the surrogate model
$A_S$ of $A$, the augmented surrogate.  Moderate gains are achieved when the number of augmentation parameters is small and the cost of sampling $A$ outside of the prediction
scenario $Q$ is much cheaper.  If the gradient of $A$ with respect to $\theta$ is only weakly dependent on the augmentation parameters, 
significant (cost-based) accuracy savings can be achieved.  

There are two fundmental assumptions in the construction of an augmented surrogate of
a $n$-dimensional predictive scenario $Q(\theta_1,\theta_2,\ldots,\theta_n)$.
\begin{enumerate}
\item There exists small number $m$ of (usually deterministic) parameters $d_1,d_2,\ldots,d_m$ that characterize
nearby $n$-dimensional predictive scenarios $\tilde{Q}_i(\theta_1,\theta_2,\ldots,\theta_n)$.
\item The cost of a sample from the nearby scenario, $C_{\tilde{Q}_i}$, is much less than the cost $C_Q$ of 
a sample from the desired prediction scenario.
\end{enumerate} 

The {\em augmented surrogate} is a surrogate model $A(\theta,d)$ constructed on training data $\{(\theta,d),\hat{Q}(\theta,d)\}$ in the $m+n$-dimensional
parameter space.   %We briefly illustrate this approach via the following numerical cartoon.

\subsubsection*{Adaptive sparse grid method}  
The classical sparse grid method is dimension agnostic\cite{bungartz2004}. All interactions
of the same order are treated equally. Often, a small subset of variables
and interactions contributes significantly to the variability of the function $Q(\theta,d)$.  If the variability in the $\theta$-dimensions is
greater than the variability in the scenario parameters($\{d_i\}$) then the overall cost of constructing the 
larger dimensional surrogate is actually less, due to the cheaper computational cost of $\tilde{Q}_i$.

We modify the 
greedy algorithm for constructing $h$-adaptive generalized sparse grid (h-GSG) in \cite{Jakeman2011LocalAD}.
The hierarchical surpluses for the current sparse grid are modified with a cost weight $W_C(d_1,d_2,\ldots,d_m)$ that approximates the relative
cost of simulating the added sparse grid point, and a $m$-dimensional distance metric that penalizes sparse grid
samples that are too far away from the prediction scenario $Q(\theta)$.  This encourages parameter exploration in $\theta_i$-dimensions at
inexpensive simulation levels, while rewarding coarse grid points that reduce the local surrogate error near the prediction scenario $Q(\theta)$.  

\subsection{Proposed UQ Study}
We will conduct a base parameter scan in $(\rho_*)$, at the mean values of the uncertain inputs.  This will verify that wedge, Eulerian versus PIC, or other factors do not impact the conclusions reached in \cite{Yas_Ido}.  This scan will also provide the first training runs for
the augmented surrogate $A$.  

Data from a range of $\rho^{-1}_*$ simulations will be obtained in the interval (100,600) will be provided by Varis Carey and other members
of the BPS Scidac team  from simulations run at NERSC and Oak Ridge facilites.  A budget of 20,000 SU at NERSC has been allocated for providing the data for this scaling study(which is already partially complete at the time of this proposal)
 and for additional simulations to train the augmented surrogate to cover the input parameter space.
The main input parameters parameterize the location, slope, and shape of the core heat source and boundary sink.  Additional parameters covering
torque will be allowed if the computational budget allows further investigation.  If NERSC resources are expended, certain (small-scale)
simulations will be conducted on CU-Denver or RMCC resources(Summit) under the supervision of Varis Carey.


 


 
